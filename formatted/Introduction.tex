\documentclass[main.tex]{subfiles} 


\begin{document}

Technology firms are highly dependent on their software engineers for innovation. Therefore human resources are an important part of their business. However a recent survey by \textit{PayScale} \cite{turnover} suggests that tech companies have the highest turover rate compared to other industries. The turnover rate measures the average tenure of the employees. One strategy for human resources to obtain employees with a lower turnover is referrals. Researchers have observed that employees who got referred by a friend or acquaintance stay longer than employees who got the job through more formal channels \cite{sixth,ten}. This paper discusses the strategy of referrals and answers the question: \emph{What are advantages and disadvantages of internal referrals as a recruitment strategy?}. \\

A survey by CareerXroads \cite{referralpercentage} observed that 27,5 \% of all job openings are filled by referrals, the biggest source of all. Although this percentage may differ geographically and per firm it is true that referrals are a common strategy to find new employees. Most companies even have bonus programs in place to  incentivise employees to refer a friend or acquaintance. These firms belief that referrals create a cheap \cite{fourth} and reliable way to find new highly productive employees. However do people really refer the best candidates in their network and how does the team or organisation change after a referral is hired? There are two levels at play here the first being the personal level and the second the organizational one. \\

Imagine there is a job opening in your team and you have multiple friends who would qualify the requirements. What does it take for you to refer a friend? This is a really important question to ask yourself, because referring a friend or acquaintance couples two worlds that are currently seperated; work and personal life. Offer money as incentive may entice you to referring someone, but would you really refer the best people in your network or does this decision depend on other factors? In this paper the personal experience of one of the author is discussed regarding referring a friend and afterwards the theory related to personal behaviour is discussed.

Another level is the organizational one; do firms benefit from employee referrals. Referrals in the same team may lead to changes in group dynamics and behaviour of others as they now see the two colleagues as an inner circle. Referrals may also damage the diversity of the employees. Research has shown that people refer people like themselves \cite{third} and therefore the diversitity within a firm (race, gender, age) may decline if referrals are predominatly used.

The next section describes the personal experience of one of the authors who recently made a (succesfull) referral and describes the upfront thought process and the change in current work situation. The second section discusses literature, observations made on referral productivity, tenure and other measurements. The main models are also discussed here. Finally the conclusions are drawn based upon personal experience and theory. It discusses shortcomings of referrals and situations where they can be beneficial.


\end{document}