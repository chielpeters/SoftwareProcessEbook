\documentclass[Main.tex]{subfiles} 


\begin{document}

In this section multiple topics are discussed on referrals. Again the analysis is made on two different levels, the personal level and the organizational level. 

\subsection*{Personal factors}

\subsubsection*{Motives: Why refer someone?}

According to literature (\cite{motives}) there are three different kinds of motivation: intrinsic, extrensic and prosocial. Intrinsic motivation refers to motivation to work for the sake of it being satisfactory or interesting. External factors such as rewards or punishment lead to extrensic motivation, motivation outside of the behaviour itself. The third and most recent kind of motivation is prosocial that is motivation to engage in work to aid other people. In the context of referrals intrinsic motivation means that employees who like there job are likely to refer others within their network. Extrensic motivation is provided by bonus referral programs although the effect of these bonuses is still unknown. People wanting to help out other people find a suitable job is the primary proscial motivation. Together these three kinds of motivation determine if an employee will make a referral or not.

%\subsubsection*{Employees refer people like themselves}

%A study performed by Taber et al. \cite{third} tried to predict if an employee was referred or come into the company on 

\subsection*{Organizational factors}

\subsubsection*{Longer tenure}
In most cases a company rather has employees that stay over a longer time than the ones that leave within a year. If an employee stays for a longer time they will gain more knowledge about the company and their job which they can use in their job. 
The results in research about the tenure, or turnover rate, in companies is inconclusive.

Caldwell and Spivey found out that the turnover rate for employees that got the job via employee referral was higher than for the ones that came in via formal methods. However, the study was performed on 1400 store clerks which have a high turnover rate no matter how they are recruited \cite{second}. In a research done by Gannon the turnover rate for employees that got the job via referrals was among the lowest, together with reemployment and referral by highschool, also very informal methods \cite{seventh}.
Research towards turnover rates was also done by Keaveney and Allan who looked at samples of business administration graduates and engineering graduates. They found out that the turnover rate for businsess administration graduates was lower when they were hired by job referral. On the other hand, it didn't really matter for the engineering graduates \cite{eleventh}.

An employee might feel the obligation to stay longer when they were hired via referral. Or they may have a better relationship with the people in the company because he/she already knows them. However, tenure is not only influenced by how a person was recruited. 

Tenure is also influenced by personality like: how ambitious is the employee. Maybe someone is always looking for a better job and grasps that oppertunity when it is given. Or maybe someone is easily bored and likes to learn new methods everytime.

Employees with a long tenure are good to have because they will know the company by heart. On the other hand, new employees may have knowledge that was not yet in the company. Even though some people will only be around for a small time, they will add value. 

\subsubsection*{Yield ratio}
The yield ratio is the proportion of new hires from the pool of applicants that was found with a certain recruitment source. A high yield ratio means that the recruitment strategy used attracted a lot of suitable applicants. Also, a high yield ratio implies that unnecessary costs were reduced. The higher the yield ratio of a strategy, the better the investment was. 
Rafaeli et al. found out that the yield ratio of the employee referrals was significantly greater than that of employment advertising \cite{fourth}. The yield ratio for employee referrals was .133 (109 new
hires out of 821 applicants) as compared to .032 (23 new hires out of 724) for the formal methods. 

The higher yield ratio can be explained by self screening of the applicants. The applicants that were recruited by employee referral have a better understanding of the what the job will be like and how the company is. They were informed by a (close) friend who already knows about the company and is able to give honest information. Therefore, the applicant has more information to make a good judgement, or self screening on whether he or she is suitable for the job. Also, the applicants already have been through an informal screening, the one there referral did. They would not be asked to apply for the job if the friend did not find them suitable in the first place. 

\subsubsection*{The costs of referrals}

Obtaining new employees can be costly process. Multiple interviews, assesments and resumees checking take quite a number of hours. Also the jobs most be advertised somewhere to notify possible job seekers or for manager position a seperate company must be hired to search for possible candidates. Referrals  must also be able to withstand the interviews and assesments, but are more likely to succeed (see previous section). The cost of referral can therefore be significantly lower compared to external sources. Rafaeli et al. (\cite{fourth}) researched the direct costs of different recruitment sources and found that referrals were the most cost efficiënt source compared to different newspaper advertisements.


\subsubsection*{Diversity in workplace}
As told in section \emph{Employees refer people like themselves} employees will most likely refer people like themselves. However, for a company it might be interesting to have a high diversity of employees in teams. 

Alfaro found that nationality diversity was positively related to team performance when project leaders regularly coordinate and monitor activities of the team members. However, when nationality diversity was low teams performed better when leaders did 
not regularly engage on these roles.This means that in case of SCRUM teams or other more agile methods the diversity in terms of different nationalities does not add value to the performance \cite{diversity}.
Not only a difference in gender or age can have an impact on team performace. According to Liang et al it is important to have a diversity in background so there is a diversity in knowledge available \cite{teamdiversity}. This difference can already exist if employees studied at different universities or have different hobbies. The more diverse the knowledge was, the better the team performed. With employee referral chances are that people with the same background (friends or relatives) will be hired. 

Results from a study of Kirnan, Farley and Geisinger indicated that females and blacks used a proportionately greater number of formal recruiting sources than their male and non-minority counterparts did \cite{tenth}. This implies that when a company wants to keep the diversity or wants to get more diversity the company should also use formal recruiting methods. Also, when applicants are referred to they are probably friends or relatives of the referrer. To keep or gain a diversity in backgrounds formal a company can use formal recruitment strategies.


\end{document}