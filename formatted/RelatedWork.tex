\documentclass[Main.tex]{subfiles} 


\begin{document}

In this section multiple topics are discussed on referrals. Again the analysis is made on two different levels, the personal level and the organizational level. 

\subsection*{Personal factors}

\subsubsection*{Motives: Why refer someone?}

According to literature (\cite{motives}) there are three different kinds of motivation: intrinsic, extrensic and prosocial. Intrinsic motivation refers to motivation to work for the sake of it being satisfactory or interesting. External factors such as rewards or punishment lead to extrensic motivation, motivation outside of the behaviour itself. The third and most recent kind of motivation is prosocial that is motivation to engage in work to aid other people. In the context of referrals intrinsic motivation means that employees who like there job are likely to refer others within their network. Extrensic motivation is provided by bonus referral programs although the effect of these bonuses is still unknown. People wanting to help out other people find a suitable job is the primary proscial motivation.

\subsubsection*{Employees refer people like themselves}


\subsection*{Organizational factors}

\subsubsection*{Longer tenure} %zwaar onder construction
In most cases a company rather has employees that stay over a longer time than the ones that leave within a year. If an employee stays for a longer time they will gain more knowledge about the company and their job which they can use in their job. 

The results in research about the tenure, or turnover rate, in companies is inconclusive.

\emph{names} found out that the turnover rate for employees that got the job via employee referral was higher than for the ones that came in via formal methods. However, the study was performed on 1400 store clerks which have a high turnover rate no matter how they are recruited \cite{second}. In a research \emph{names} did the turnover rate for employees that got the job via referrals was among the lowest, together with reemployment and referral by highschool, also very informal methods \cite{seventh}.
Research towards turnover rates was also done by \emph{names} who looked at samples of business administration graduates and engineering graduates. They found out that the turnover rate for businsess administration graduates was lower when they were hired by job referral. On the other hand, it didn't really matter for the engineering graduates \cite{eleventh}.

%En dit nog even verwerken morgen
%Not only is turnover influenced 
%by the perceived desirability of a job change, it is also determined 
%by the perceived ease of movement \cite{eleventh}
%\emph{There are several possible explanations for the association 
%between turnover and recruiting method characterizing the 
%business subsample. Individuals using alternative job sources may 
%differ with respect to their inherent abilities. It is possible that 
%higher turnover among those using formal methods, primarily 
%college placement offices in this sample, occurs because they are 
%more qualified than their counterparts using other job sources. 
%As a result, more employment opportunities might be available 
%to them. They might also be more aggressive, have higher needs for 
%achievement, or make greater efforts to seek out other job oppor- 
%tunities. Alternatively, the lower turnover among members of the 
%business subsample using personal referrals could be a function 
%of favorable interpersonal relations. Perhaps peer group relations 
%and superior-subordinate relations are better when one uses this 
%method to obtain employment. Also, it is possible that an 
%individual obtaining a job this way feels an obligation to stay with 
%a job longer than is the case with other recruiting methods.}\\ 

%\emph{Within the engineering subsample, the proportion changing 
%employers was approximately the same for individuals using 
%personal referrals and formal methods. However, those using direct 
%applications were less likely to change employers than those 
%utilizing the other job sources. It is possible that individuals 
%utilizing direct applications have made a greater investment in 
%time and effort to obtain their initial positions. As a result, they 
%may be less likely to leave that employer

\subsubsection*{Yield ratio}
The yield ratio is the proportion of new hires from the pool of applicants that was found with a certain recruitment source. A high yield ratio means that the recruitment strategy used attracted a lot of suitable applicants. Also, a high yield ratio implies that unnecessary costs were reduced. The higher the yield ratio of a strategy, the better the investment was. 
Rafaeli et al. found out that the yield ratio of the employee referrals was significantly greater than that of employment advertising \cite{fourth}. The yield ratio for employee referrals was .133 (109 new
hires out of 821 applicants) as compared to .032 (23 new hires out of 724) for the formal methods. 

The higher yield ratio can be explained by self screening of the applicants. The applicants that were recruited by employee referral have a better understanding of the what the job will be like and how the company is. They were informed by a (close) friend who probably already knows a lot about the company and give honest information. Therefore, the applicant has more information to make a good judgement, or self screening on whether he or she is suitable for the job. Also, the applicants already have been through an informal screening, the one there referral did. They would not be asked to apply for the job if the friend did not find them suitable in the first place. 

\subsubsection*{The costs of referrals}

\subsubsection*{Diversity in workplace}
%find a paper about diversity in software teams
%make a reference to subsubsection employees refer people like themselves
% age and stuff?



\end{document}