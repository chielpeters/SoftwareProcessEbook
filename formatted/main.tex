%----------------------------------------------------------------------------------------
%	PACKAGES AND OTHER DOCUMENT CONFIGURATIONS
%----------------------------------------------------------------------------------------

\documentclass[a4paper, 11pt]{article} % Font size (can be 10pt, 11pt or 12pt) and paper size (remove a4paper for US letter paper)

\usepackage{fullpage}
\usepackage[protrusion=true,expansion=true]{microtype} % Better typography
\usepackage{graphicx} % Required for including pictures
\usepackage{wrapfig} % Allows in-line images
\usepackage{subfiles}
\usepackage{hyperref}
\usepackage{mathpazo} % Use the Palatino font
\usepackage[T1]{fontenc} % Required for accented characters
\linespread{1.05} % Change line spacing here, Palatino benefits from a slight increase by default

\makeatletter
\renewcommand\@biblabel[1]{\textbf{#1.}} % Change the square brackets for each bibliography item from '[1]' to '1.'
\renewcommand{\@listI}{\itemsep=0pt} % Reduce the space between items in the itemize and enumerate environments and the bibliography

\renewcommand{\maketitle}{ % Customize the title - do not edit title and author name here, see the TITLE block below
\begin{flushright} % Right align
{\LARGE\@title} % Increase the font size of the title

\vspace{50pt} % Some vertical space between the title and author name

{\large\@author} % Author name
\\\@date % Date

\vspace{20pt} % Some vertical space between the author block and abstract
\end{flushright}
}

\usepackage{epigraph}

% \epigraphsize{\small}% Default
\setlength\epigraphwidth{12cm}
\setlength\epigraphrule{0pt}

\usepackage{etoolbox}

\makeatletter
\patchcmd{\epigraph}{\@epitext{#1}}{\itshape\@epitext{#1}}{}{}
\makeatother
\setlength\parindent{0cm}

%----------------------------------------------------------------------------------------
%	TITLE
%----------------------------------------------------------------------------------------
\title{\textbf{Human Resources}\\ % Title
The value of referrals} % Subtitle

\author{\textsc{Cindy Berghuizen \& \\ Chiel Peters} % Author
\\{\textit{University of Amsterdam}}} % Institution

\date{\today} % Date

%----------------------------------------------------------------------------------------

\begin{document}

\maketitle % Print the title section

%----------------------------------------------------------------------------------------
%	ABSTRACT AND KEYWORDS
%----------------------------------------------------------------------------------------

%\renewcommand{\abstractname}{Summary} % Uncomment to change the name of the abstract to something else

\epigraph{``Human resources are like natural resources; they're often buried deep. You have to go looking for them, they're not just lying around on the surface. You have to create the circumstances where they show themselves."}{--- \textup{Ken Robinson}}

%\begin{abstract}
%Human resources are very important for technology firms as most of the knowledge about the software systems is tacit. Learning time for new systems grow fast as the systems become more and more complex. However a recent survey showed that technology firms have the highest turnover rate\cite{turnover}. Although this can be contributed to numerous facts and even be seem as a positive, this paper will reach one possible human resource strategy to obtain more qualified engineers, referrals. \\
%
%Referrals are currently the biggest source for job openings \cite{referralpercentage} and observation have reported a longer tenure for these referred employees. One possible explanation is that they are more informed about the job and therefore can better asses there preferences. Another benefit is that referrals are more likely to get the job and therefore more valuable to interview because employees already performed a pre-screening action. There are however also negative sides of referrals, research has shown that people refer people like themselves \cite{second} and therefore workplace diversity is greatly diminished.
%\end{abstract}


\vspace{30pt} % Some vertical space between the abstract and first section

%----------------------------------------------------------------------------------------
%	ESSAY BODY
%----------------------------------------------------------------------------------------

\section*{Introduction}
\subfile{Introduction}

\section*{Personal Experience}
\subfile{personalExp}

\section*{Related work}
\subfile{RelatedWork}

\section*{Conclusion}
\subfile{Conclusion}


\section*{Annotated Bibliography}
\subfile{Annotated}

%----------------------------------------------------------------------------------------
%	BIBLIOGRAPHY
%----------------------------------------------------------------------------------------

\bibliographystyle{unsrt}

\bibliography{sample}

%----------------------------------------------------------------------------------------

\end{document}
