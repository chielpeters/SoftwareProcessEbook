\documentclass[Main.tex]{subfiles} 


\begin{document}

In this section the personal experience of one of the authors and thoughts related to the value of referrals are stated. This section was written before any of the relevant theory and models were researched to avoid any biases in our observations. The effects of referrals are measured on two different levels, the personal level that is the effects on a person making the referral and the organizational level which measures the effects based on the organizational as a whole. To give some contexts to the personal experience first the background of the author is discussed after which the effects on the two levels are mentioned.

\subsection*{Background}

 Since three years I have been working at small risk consultancy firm with approximately ten employees. All the employees have the same background education and are working in a very specific field. The projects are done in very small groups which consist of one to three persons. Being so small means the communication lines are short and a lot of situations are discussed with the whole firm. Recently a new student position opened up and sollications poured in.  Now one of my college friends which I now know for about ten years was searching for such a position. From the ten employees four of those came directly from referrals made by the employees themselves so the culture of referrals was there. After a few moments of thought I decided to go through and later on he was hired for the position. In the next section this decision is elaborated upon and the thought process is carefully described. \\

Besides the situation I have asked one of the people close to me to describe my personality. This section is a personal experience and can very much differ between persons with different personalities. The following description of me was made: \\

\emph{"He is a extrovert person, very social and likes to be with other people. However when it is needed he can close himself from his environment and be and work alone. Challenges are key as he is very open to new experiences. Conscientiousness applies to his personality, if he picks up a task he will put as much effort as it takes to let the task succeed to his standards. He has a very strong opinion on matters, knows what he wants or likes and is not afraid to let this know to the people around him".}

\subsection*{Personal Factors}

There were multiple people within in my network to refer for this job, so why do you choose that particulair person? This is a really hard question and depends  on three things for me: do I want to work with this person, does he/she fit in the team and ultimately how likely is this person to succeed?. These are the main questions I asked myself while choosing who to refer. In the next paragraphs I will further explain each of the ideas behind each of these questions.\\

While asking do I want to work with this person, you assess if working together seems enjoyable to you but most importantly you ask yourself if you want to break the seperation between work and private life. This was very important issue for me especially since in a small firm were you will be working side by side. I was a bit hesitant at first, because I value this seperation very much. However it is actually good to break this seperation because now I have someone to discuss work with in my personal life who understands the situation and knows the people. This flow of information from work to personal is a good one, the other way around is usually perceived as scary. In my situation no personal information ever comes up at work that I do not want to get there. If something troubling would occur in my personal environment than I could always ask my referral not to bring it up at work. \\

Another important point for me was how well does he/she fit in the team. If any possible irritations break out I do not want to have to choose sides. Although you never really can tell how this is going to work out, it is more or less judged upon with gut feeling. In my case my gut feeling said that this was no problem and luckely I was right and he fitted right in.\\

Fitting in is nice, but it is still work and therefore performance matters. If your referral fails than this reflects poorly on you. Ultimately it is the firms responsibility to check if your referral meets their criteria however in my situation blind trust was placed upon me. Knowing this also leads to changed behaviour compared to other collegeaus. I am more likely to help my referral in his projects to make him succeed, because his performance in the beginning feels your responsibility.

\subsection*{Organizational Factors}

As stated in the introduction a lot of firms have bonus referral programs which reward employees to refer friends or acquaintances. Even with the bonus referral programs employee referrals cost less money than for example newspaper advertisements or using agencies. Therefore it seems there are only benefits to referrals and no downsides. These benefits and possible downsides are discussed in this section.\\

As mentioned in the background information referrals are highly encouraged in the firm I am working for. So why are referrals are highly valued at my firm? I think this depends one key thing: \emph{trust}. Being small and highly specific means company image is very important. Therefore employees receive a lot of trust to perform their tasks well at clients as no one is watching over their shoulder. For the partners of the firm it is easier to trust someone who knows is trusted by someone within the firm already than an outsider. \\

There are however two downsides the firm considers with referrals: islands and performance. In this paragraph the islands issue is discussed. Too much referrals from the same network can create islands within the workplace. Since this was my first referral this was not a problem. However I do feel it is true that some collegeaus see both of you as a unit with the same opinion however mostly this is not the case. \\

Another considered downside was performance. In my case I would say it did not hurt performance. It does lead to more stories about past experiences, but this does not hurt performance. I think this really depends on the personality of both the one referring and the referral. I'm comfortable zoning of when real work needs to be done. However referrals do increase small talk within the team. 


\end{document}