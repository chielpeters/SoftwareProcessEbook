\documentclass[Main.tex]{subfiles} 


\begin{document}

In this section our personal experience and thoughts related to the value of referrals are stated. This section was written before any of the relevant theory and models were researched to avoid any biases in our observations. The effects of referrals are measured on two different levels, the personal level that is the effects on a person making the referral and the organizational level which measures the effects based on the organizational as a whole. To give some contexts to our personal experience first our backgrounds are discussed after which the effects on the two levels are mentioned. In this section the experience of one of the authors is stated. 

\subsection*{Background}

 Since three years I have been working at small risk consultancy firm with approximately ten employees.All the employees have the same background education and are working in a very specific field. The projects are done in very small groups which consist of one to three persons. Being so small means the communication lines are short and a lot of situations are discussed with the whole firm. Recently a new student position opened up and sollications poured in.  Now one of my college friends which I now know for about ten years was searching for such a position. From the ten employees four of those came directly from referrals made by the employees themselves so the culture of referrals was there. After a few moments of thought I decided to go through and later on he was hired for the position. In the next section this decision is elaborated upon and the thought process is carefully described. \\

Besides the situation I have asked one of the people close to me to describe my personality. This section is a personal experience and can very much differ between person. The following description of me was made: \\

\emph{"He is a extrovert person, very social and likes to be with other people. However when it is needed he can close himself from his environment and be and work alone. Challenges are key as he is very open to new experiences. Conscientiousness applies to his personality, if he picks up a task he will put as much effort as it takes to let the task succeed to his standards. He has a very strong opinion on matters, knows what he wants or likes and is not affraid to let this know to the people around him".}

\subsection*{Personal Factors}

This section covers the personal factors involved into referrals. Referring a friend or acquaintance often involves numerous trade offs and carefull examination. Each of the following subsections contains a question or thought process that lead to making the referral decision.


\subsubsection*{What dangers do I run when I stick my stick my neck out for this person?}

If the person being referred fails than this reflects on you. Ultimately it is the firms responsibility to check if your referral meets their criteria however in my situation blind trust was placed upon me. Knowing this also leads to changed behaviour compared to other collegeaus. I am more likely to help my referral in his projects to make him succeed, because his performance in the beginning feels your responsibility.

\subsubsection*{What are the consequences of breaking the seperation of business and private life?}

This was very important issue especially in a small firm were you will be working side by side. I was a bit hesitant at first, because I value this seperation very much. However I currently like it very much because I have someone to discuss work with in my personal life who understands the situation and knows the people. This flow of information from work to personal is a good one, the other way around is usually perceived as scary. In my situation no personal information ever comes up at work that I do not want to get there. If something troubling would occur in your personal environment than I could always ask my referral not to bring it up at work. 

\subsection*{Organizational Factors}

In this section the organizational factors of referrals are discussed. As stated in the introduction a lot of firms have bonus referral programs which reward employees to refer friends or acquaintances. Even with the bonus referral programs employee referrals cost less money than for example newspaper advertisements or using agencies. Therefore it seems there are only benefits to referrals and no downsides. These benefits and possible downsides are discussed in each of the subsections.

\subsubsection*{What are the main differences between formal advertisements and referrals?}

Referrals are highly valued at the firm. Being small and highly specific means company image is very important. Therefore employees receive a lot of trust to perform their tasks well at clients. For the partners of the firm it is easier to trust someone who knows is trusted by someone within the firm already than an outsider. 

\subsubsection*{Do referrals create islands in the team?}

No, I do not beleive islands are the right words. It is true that some collegeaus see both of you as a unit with the same opionins however mostly this is not the case. 

\subsubsection*{Does the personal atmosphere hurt the performance?}

In my case I would say not. It does lead to more stories about past experiences, but this does not hurt performance. I think this really depends on the personality of both the one referring and the referral. I'm comfortable zoning of when real work needs to be done. However I also believe that it does increase small talk within the team. 


\end{document}