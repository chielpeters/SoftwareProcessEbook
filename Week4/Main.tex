\documentclass{article}
\usepackage{fullpage}
\usepackage{color}
\usepackage{subfiles}
\usepackage{url}


\renewcommand{\arraystretch}{1.5}
\begin{document}
\title{Human Resources: The Value Of Intangible Assets}
\author{Chiel Peters \& Cindy Berghuizen}
\date{08-02-2013}
\maketitle

\setlength\parindent{0pt}

Technology firms are highly dependent on their software engineers for innovation or/and the implementation of innovation. Therefore human resources is an important part of their business as most there employees are a valuable intangible asset. However a recent survey by \textit{PayScale} \cite{turnover} suggests that tech companies have the highest turover rate compared to other industries. The turnover rate measures the average tenure of the employees. Even more surprisingly is that succesfull companies such as Amazon and Google rank respectively second and fourth among the list of highest turnovers. With all the employee benefits Google is offering such as free legal services, lunches, entertainment areas and learning possibilities you could wonder why are people leaving? In this chapter the value of human resources and good recruitment are discussed. In related work existing research into recruiting for technology firms are covered\footnote{A specific area within recruiting will be discussed which we will decide upon in the beginning of next week}.




\begin{thebibliography}{9}

\bibitem{turnover} Tech companies have highest turnover rate, \url{http://www.techrepublic.com/blog/career-management/tech-companies-have-highest-turnover-rate/#}, 25-02-2014

\bibitem{benefits} Benefits - Google Jobs, \url{https://www.google.nl/about/jobs/lifeatgoogle/benefits/}, 25-02-2014

\bibitem{first} Colomo-Palacios, R., Tovar-Caro, E., García-Crespo, Á., \& Gómez-Berbís, J. M. (2010). \textit{Identifying Technical Competences of IT Professionals: The Case of Software Engineers}. International Journal of Human Capital and Information Technology Professionals (IJHCITP).

\bibitem{second} Swart, J., Kinnie, N. (2006). \textit{Sharing knowledge in knowledge-intensive firms}. Human Resource Management Journal, Vol. 13, 60 - 75

\bibitem{third} HR-XML, Global HR Interoperability Standards, \url{http://www.hr-xml.org/?page=About}, 24-02-2014

\bibitem{fourth} Huselid, M., Jackson, S., Schuler, R. (1997). \textit{Technical and Strategic Human Resources Management Effectiveness as Determinants of Firm Performance}. Academy of Management Journal, Vol. 40, 171-188.

\bibitem{fifth} Gardner, T. (2005). \textit{Interfirm Competition for Human Resources: Evidence From the Software Industry}. Academy of Management Journal, Vol. 48, 237-256.

\bibitem{sixth} Chen-Fu Chien, Li-Fei Chen (2008).\textit{ Data mining to improve personnel selection and enhance human capital: A case study in high-technology industry}. Expert Systems with Applications, Vol. 34, 280-290

\bibitem{seventh} Paul, A., Anantharaman, R. (2004). \textit{Influence of HRM practices on organizational commitment: A study among software professionals in India}. Human Resource Development Quarterly, Vol. 15, 77-88.


\end{thebibliography}


\end{document}
